\RequirePackage{pdfmanagement}
\documentclass[a4paper]{l3doc}

\usepackage{microtype}
\usepackage{parskip}
\usepackage{enumitem}
\usepackage[font=small, skip=6pt]{caption}
\usepackage{tabularray}
\UseTblrLibrary{booktabs}
\usepackage{array}
\usepackage{booktabs}
\usepackage{listings}
\usepackage{afesbrand}
\usetikzlibrary{matrix}

\setlist[description]{font=\ttfamily}

\lstdefinelanguage{expl3}{
  alsoletter={:<>},
  morekeywords={
    \cs_new_protected:Nn, \str_set:Ne, \str_if_eq:VnT,
    \__afesbrand_logo_<name>_<variant>:nnn, \path, \l_tmpa_str,
    \AFESBrandFontHeadOne, \AFESBrandFontHeadTwo, \AFESBrandFontHeadThree,
    \AFESBrandFontBody, \AFESBrandIcon, \AFESBrandHeroGraphic, \AFESBrandMark,
    \AFESBrandUniMeetJesus, \AFESBrandWaveEffect, \AFESBrandFrameDevice,
    \AFESBrandGridSystem, \AFESBrandIllustration, \AFESBrandFontSelect,
    \AFESBrandEmailSignature, \AFESBrandSetColours, \centering, \color, \huge,
    \colorbox, \textcolor, \raggedright, \quad, \node, \begin, \end
  },
  sensitive=true,
  morecomment=[l]\%
}

\lstset{
  language=expl3,
  basicstyle=\ttfamily\small,
  keywordstyle=\color{royalblue},
  commentstyle=\itshape\color{black!70},
  columns=fullflexible,
  keepspaces=true,
  upquote=true,
  belowskip=-\medskipamount
}

\setcounter{tocdepth}{2}

\begin{document}

\title{The \pkg{afesbrand} Package}
\author{David Purton\thanks{Email: \url{david.purton@afes.org.au}}}
\date{2026/02/25 v1.2a}

\maketitle

\begin{abstract}
  The \pkg{afesbrand} package provides a set of macros for producing
  components of the AFES brand guidelines.
\end{abstract}

\begingroup
\renewcommand\baselinestretch{0.9}\selectfont
\tableofcontents
\endgroup

\clearpage

\section{Style Usage}

Include the package |\usepackage{afesbrand}|.

\begin{function}{\AFESBrandSetOptions}
  \begin{syntax}
    \marg{options}
  \end{syntax}
  Set package global \meta{options}. Options set using
  \cs{AFESBrandSetOptions} need a prefix outlined below in each section.
\end{function}

\section{Brand Basics}

\subsection{Logos}

\begin{function}{\AFESBrandLogo}
  \begin{syntax}
    \oarg{options}\marg{logo}
  \end{syntax}
  Print a logo.
\end{function}

The following values of \meta{logo} are supported:

\begin{description}
  \item[brandmark] Prints the tag line logo.
  \item[focus] Prints the FOCUS logo.
  \item[national] Prints the national logo.
  \item[tagline] Prints the tag line logo.
\end{description}

The following \meta{options} are supported:

\begin{description}
  \item[bgcolour] Set the background colour of the logo. Default: |none|
  \item[colour] Set the colour of the logo.
  \item[height] Set the height of the logo (including any |padding|). If both
    |height| and |width| are specified the logo will be scaled to fit within
    these dimensions.
  \item[padding] Set the padding around the logo. The following values are
    supported:
    \begin{description}
      \item[anchored] The logo is horizontally centred with padding equal to
        twice the height of the top of the cross and with no padding on the
        bottom.

        \AFESBrandLogo[bgcolour=neutral, height=1.3cm, padding=anchored]{national}

      \item[centred] The logo is centred with padding equal to twice the
        height of the top of the cross. This is the default.

        \AFESBrandLogo[bgcolour=neutral, height=1.5cm, padding=centred]{national}

      \item[none] The logo is printed without padding.

        \AFESBrandLogo[bgcolour=neutral, height=1cm, padding=none]{national}
    \end{description}
  \item[variant] Print a logo variant. See below for defined values of
    |variant| for each supported \meta{logo}.
  \item[width] Set the width of the logo (including any |padding|). If both
    |height| and |width| are specified the logo will be scaled to fit within
    these dimensions.
\end{description}

These options can also be set using \cs{AFESBrandSetOptions} and the prefix
|logo/|.

The defined combinations of \meta{logo} and \meta{variant} are shown in Table
\ref{logos}.

\begin{table}
  \centering
  \caption{Defined \meta{logo} and \meta{variant} combinations}
  \label{logos}
  \AFESBrandSetOptions{logo/height=1cm, logo/padding=none}
  \begin{tblr}{stretch=0, colspec={lQ[h,c]Q[h,c]Q[h,c]Q[h,c]}}
    \toprule
    \meta{variant}/\meta{logo} & \texttt{brandmark} & \texttt{focus} &
    \texttt{national} & \texttt{tagline} \\
    \midrule
    \texttt{primary} & \AFESBrandLogo{brandmark} & \AFESBrandLogo{focus} &
    \AFESBrandLogo{national} & \AFESBrandLogo{tagline} \\
    \texttt{primarytagline} & &
    \AFESBrandLogo[variant=primarytagline]{focus} &
    \AFESBrandLogo[variant=primarytagline]{national} & \\
    \texttt{secondary} & & & & \AFESBrandLogo[variant=secondary,
    height=0.75cm]{tagline} \\
    \texttt{vertical} & & \AFESBrandLogo[variant=vertical, height=0pt,
    width=0.85cm]{focus} & \AFESBrandLogo[variant=vertical, height=0pt,
    width=0.85cm]{national} & \\
    \texttt{verticaltagline} & & \AFESBrandLogo[variant=verticaltagline,
    height=0pt, width=0.85cm]{focus} &
    \AFESBrandLogo[variant=verticaltagline, height=0pt,
    width=0.85cm]{national} & \\
    \bottomrule
  \end{tblr}
\end{table}

\subsubsection{Adding New Logos}

New logos can be added by adding an Expl3 function of the format shown in the
template in Listing \ref{newlogos}. |<name>| and |<variant>| correspond to
the main argument and |variant| option of the \cs{AFESBrandLogo} function. You
can define as many or few variants as you need, but at least |variant=primary|
must be defined. The \cs{path} macros are \pkg{tikz} paths. If no unit us
specified for coordinates it is assumed to be |cm|.

\begin{lstlisting}[float, caption={Template for adding new logos}, label=newlogos]
\cs_new_protected:Nn \__afesbrand_logo_<name>_<variant>:nnn
  {
    \str_set:Ne \l_tmpa_str {#1}
    % Paths assume a default unit of cm
    \str_if_eq:VnT \l_tmpa_str { anchored }
      {
        % path for the anchored bounding box
        \path[fill=#2] ...;
      }
    \str_if_eq:VnT \l_tmpa_str { centred }
      {
        % path for centred bounding box
        \path[fill=#2] ...;
      }
    \str_if_eq:VnT \l_tmpa_str { none }
      {
        % path for the unpadded bounding box
        \path[fill=#2] ...;
      }
    % path for logo
    \path[fill=#3] ...;
  }
\end{lstlisting}

Save your new logos in a file called |afesbrand_logos_<affiliate>.def| and
they can then be loaded use the |\AFESBrandLogoLoad| function.

\begin{function}{\AFESBrandLogoLoad}
  \begin{syntax}
    \marg{affiliate}
  \end{syntax}
  Load a set of logos from a file called |afesbrand_logos_<affiliate>.def|.
\end{function}


\subsection{Colour Palette}

The AFES brand colours are defined using the \pkg{xcolor} package. They are
defined in both RGB and CMYK so if you load \pkg{xcolor} with the |cmyk|
option \emph{before} loading \pkg{afesbrand}, they will be defined in CMYK.
See Figure \ref{palette}.

\begin{figure}
  \centering
  \catcode`\|=12
  \begin{tikzpicture}
    \matrix
      [matrix of nodes, outer sep=0pt, inner sep=0pt,
          nodes={font=\ttfamily, minimum size=3cm, align=center, text
                 width=3cm, anchor=center}]
        {
          |[fill=dustyblue]| dustyblue \#9AD9E9 &
          |[fill=slatenavy, text=white]| slatenavy \#242C37 &
          |[fill=lemonyellow]| lemonyellow \#FAF5B2 &
          |[fill=naturalgrey]| naturalgrey \#E6E6DF \\
          |[fill=huntergreen, text=white]| huntergreen \#34613B &
          |[fill=royalblue, text=white]| royalblue \#006CC5 &
          |[fill=sunnyyellow]| sunnyyellow \#EFD954 &
          |[fill=tangerineorange, text=white]| tangerineorange \#FF6622 \\
          |[fill=sagegreen]| sagegreen \#6BB07D &
          |[fill=digitalpink]| digitalpink \#FFBAFF &
          |[fill=mustardyellow]| mustardyellow \#E8AD21 &
          |[fill=lightorange]| lightorange \#FFC88D \\
        };
  \end{tikzpicture}
  \caption{AFES colour palette}\label{palette}
\end{figure}

\begin{function}{\AFESBrandSetColours}
  \begin{syntax}
    \marg{options}
  \end{syntax}
  Set the default \emph{neutral}, \emph{hero} and \emph{accent} colours.
\end{function}

The following \meta{options} are supported and shown in Figure
\ref{defaultcolours}.

\begin{description}
  \item[neutral] Set the \emph{neutral} colour. Default: |naturalgrey|
  \item[hero] Set the \emph{hero} colour. Default: |huntergreen|
  \item[accent] Set the \emph{accent} colour. Default: |dustyblue|
\end{description}

\begin{figure}
  \centering
  \catcode`\|=12
  \begin{tikzpicture}
    \matrix
      [matrix of nodes, outer sep=0pt, inner sep=0pt,
          nodes={font=\ttfamily, minimum size=3cm, anchor=center}]
        {
          |[fill=neutral]| neutral &
          |[fill=hero, text=white]| hero &
          |[fill=accent]| accent \\
        };
  \end{tikzpicture}
  \caption{Default \texttt{neutral}, \texttt{hero} and \texttt{accent}
  colours}\label{defaultcolours}
\end{figure}

These options can also be set using \cs{AFESBrandSetOptions} and the prefix
|colour/|.

\subsection{Typography}

\begin{function}{\AFESBrandFontSelect}
  \begin{syntax}
    \marg{options}
  \end{syntax}
  Select a font with the following supported \meta{options}:
\end{function}

\begin{description}
  \item[family] Either |livvic| or |poetsenone|
  \item[series] A supported NFSS font series.

    For |poetsenone| only |m| is available
    ({\AFESBrandFontSelect{family=poetsenone}Poetsen One}).  The NFSS series
    and corresponding weights for the |livvic| family are shown in Table
    \ref{livvic}.

    \begin{table}
      \centering
      \caption{NFSS series and weights for \texttt{livvic} font family}
      \label{livvic}
      \begin{tabular}{ll}
        \toprule
        \textbf{NFSS Series} & \textbf{Font Weight} \\
        \midrule
        |ul| & {\AFESBrandFontSelect{family=livvic,series=ul}Livvic Thin} \\
        |el| & {\AFESBrandFontSelect{family=livvic,series=el}Livvic ExtraLight} \\
        |l| & {\AFESBrandFontSelect{family=livvic,series=l}Livvic Light} \\
        |sl| & {\AFESBrandFontSelect{family=livvic,series=sl}Livvic Regular} \\
        |m| & {\AFESBrandFontSelect{family=livvic,series=m}Livvic Medium} \\
        |sb| & {\AFESBrandFontSelect{family=livvic,series=sb}Livvic SemiBold} \\
        |b| & {\AFESBrandFontSelect{family=livvic,series=b}Livvic Bold} \\
        |ub| & {\AFESBrandFontSelect{family=livvic,series=ub}Livvic Black} \\
        \bottomrule
      \end{tabular}
    \end{table}

  \item[shape] A supported font shape. For |poetsenone| only |n| is available.
    For |livvic|, |n| and |it| are available. Unfortunately |livvic| does not
    support the |sc| shape.
  \item[letterspace] The letter spacing which is passed to \pkg{fontspec}.
    See the \pkg{fontspec} manual for details.

    This defaults to |0| for |livvic| and |-5| for |poetsenone|.
  \item[linespace] The line spacing. This is a numeric argument which when
    multiplied by the font |size| will give the \cs{baselineskip}. So for a
    10pt font, |linespace=1.2| will give a \cs{baselineskip} of 12pt.
  \item[relsize] A numeric value which specifies the font |size| as a multiple
    of the current font size. This takes precedence over the |size| option.
  \item[size] The size of the font. This can either be a numeric value
    indicating the point size or a standard \LaTeX{} font size macro
    (\cs{footnotesize}, \cs{small}, \cs{normalsize}, \cs{large}, etc.).
  \item[wordspace] A multiplier for the standard word spacing.
\end{description}

These options can also be set using \cs{AFESBrandSetOptions} and the prefix
|font/|.

\textbf{Note:} You will need to download and install the fonts from:
\begin{itemize}
  \item \url{https://fonts.google.com/specimen/Livvic} and
  \item \url{https://fonts.google.com/specimen/Poetsen+One}
\end{itemize}

\begin{function}{\AFESBrandFontHeadOne}
  Select the style specified level 1 heading font.
\end{function}

\begin{lstlisting}
\AFESBrandFontHeadOne\textcolor{hero}{%
  MEET \AFESBrandIcon[bgcolour=sunnyyellow, colour=slatenavy]{cross}
  JESUS\\ SEE THE BIGGER\\ PICTURE
  \AFESBrandIcon[bgcolour=tangerineorange, colour=slatenavy]{sun}}
\end{lstlisting}

{\AFESBrandFontHeadOne \textcolor{hero}{MEET
 \AFESBrandIcon[bgcolour=sunnyyellow, colour=slatenavy]{cross} JESUS\\ SEE THE
 BIGGER\\ PICTURE \AFESBrandIcon[bgcolour=tangerineorange,
  colour=slatenavy]{sun}}\par}

\begin{function}{\AFESBrandFontHeadTwo}
  Select the style specified level 2 heading font.
\end{function}
 
\begin{lstlisting}
\AFESBrandFontHeadTwo A little bit about us
\end{lstlisting}

{\AFESBrandFontHeadTwo A little bit about us\par}

\begin{function}{\AFESBrandFontHeadThree}
  Select the style specified level 3 heading font.
\end{function}

\begin{lstlisting}
\AFESBrandFontHeadThree\textcolor{tangerineorange}{ABOUT US}
\end{lstlisting}

\medskip

{\AFESBrandFontHeadThree \textcolor{tangerineorange}{ABOUT US}\par}

\begin{function}{\AFESBrandFontBody}
  Select the style specified body font.
\end{function}

\begin{lstlisting}
\AFESBrandFontBody\raggedright What if uni...
\end{lstlisting}

{\AFESBrandFontBody\raggedright What if uni isn't just about your degree, but
 discovering something bigger? Your plans, your future – they're not meant to
 be detached from the one who made you. Joining an AFES group is like finding
 solid ground. We'll help you lift your eyes up from uni's daily hustle to see
 Jesus – the one who gives meaning to it all.\par}

\begin{function}{\AFESBrandUniMeetJesus}
  Print the AFES tag line with the raised comma.
\end{function}

\begin{lstlisting}
\textcolor{hero}{\AFESBrandUniMeetJesus}
\end{lstlisting}

\medskip

{\centering\textcolor{hero}{\AFESBrandUniMeetJesus}\par}

\section{Graphic Elements}

\subsection{Brand Icons}

\begin{function}{\AFESBrandIcon}
  \begin{syntax}
    \oarg{options}\marg{icon}
  \end{syntax}
  Print a one of the standard AFES icons using the specified \meta{options}.
\end{function}

See Table \ref{brandicons} for supported values of \meta{icon}.

\begin{table}
  \centering
  \caption{Supported brand icons}\label{brandicons}
  \AFESBrandSetOptions{icon/size=5mm}
  \renewcommand{\arraystretch}{1.5}
  \begin{tabular}{>{\ttfamily}ll>{\ttfamily}ll}
    \toprule
    arrow & \AFESBrandIcon{arrow} & heart & \AFESBrandIcon{heart} \\
    book & \AFESBrandIcon{book} & lens & \AFESBrandIcon{lens} \\
    cross & \AFESBrandIcon{cross} & question & \AFESBrandIcon{question} \\
    globe & \AFESBrandIcon{globe} & smile & \AFESBrandIcon{smile} \\
    hand & \AFESBrandIcon{hand} & sun & \AFESBrandIcon{sun} \\
    \bottomrule
  \end{tabular}
\end{table}

The following \meta{options} are supported:

\begin{description}
  \item[bgcolour] The background colour. Default: |neutral|
  \item[colour] The icon colour. Default: |hero|
  \item[size] The size of the icon as a dimension expression. If not
    specified, the icon will be sized to the current text size.
\end{description}

These options can also be set using \cs{AFESBrandSetOptions} and the prefix
|icon/|.

\subsection{Utility Icons}

\begin{function}{\AFESBrandIconUtility}
  \begin{syntax}
    \oarg{options}\marg{icon}
  \end{syntax}
  Print a utility icon using the specified \meta{options}.
\end{function}

Icons are taken from the Phospor icon family. You'll need to download and
install |Phosphor-Fill.ttf| from \url{https://phosphoricons.com/}.

Look up icons and names from the website and pass the name to
\cs{AFESBrandIconUtility}. See Table \ref{utilityicons} for some example
utility icons.

\begin{table}
  \centering
  \caption{Example utility icons}\label{utilityicons}
  \begin{tabular}{>{\ttfamily}ll>{\ttfamily}ll>{\ttfamily}ll>{\ttfamily}ll}
    \toprule
     church & \AFESBrandIconUtility{church} &
     acorn & \AFESBrandIconUtility{acorn} &
     heart & \AFESBrandIconUtility{heart} &
     balloon & \AFESBrandIconUtility{balloon} \\
     cactus & \AFESBrandIconUtility{cactus} &
     butterfly & \AFESBrandIconUtility{butterfly} &
     crown & \AFESBrandIconUtility{crown} &
     bus & \AFESBrandIconUtility{bus} \\
     heart & \AFESBrandIconUtility{heart} &
     city & \AFESBrandIconUtility{city} &
     island & \AFESBrandIconUtility{island} &
     books & \AFESBrandIconUtility{books} \\
     \bottomrule
  \end{tabular}
\end{table}

The following \meta{options} are supported:

\begin{description}
  \item[colour] The icon colour. Default: \emph{Current colour}
  \item[relsize] A numeric value which specifies the font |size| as a multiple
    of the current font size. This takes precedence over the |size| option.
  \item[size] The size of the font. This can either be a numeric value
    indicating the point size or a standard \LaTeX{} font size macro
    (\cs{footnotesize}, \cs{small}, \cs{normalsize}, \cs{large}, etc.).
\end{description}

These options can also be set using \cs{AFESBrandSetOptions} and the prefix
|iconutility/|.

\subsection{Illustrations}

\begin{function}{\AFESBrandIllustration}
  \begin{syntax}
    \oarg{options}\marg{illustration}
  \end{syntax}
  Print an illustration using the specified \meta{options}.
\end{function}

Table \ref{illustrations} shows values for \meta{illustration} defined by
default.

\begin{table}
  \centering
  \caption{Supported illustrations}\label{illustrations}
  \AFESBrandSetOptions{illustration/height=1.3cm}
  \renewcommand{\arraystretch}{1.5}
  \begin{tabular}{>{\ttfamily}ll>{\ttfamily}ll}
    \toprule
    croissant & \raisebox{-0.5\height}{\AFESBrandIllustration{croissant}} &
    lamb & \raisebox{-0.5\height}{\AFESBrandIllustration{lamb}} \\
    lighthouse & \raisebox{-0.5\height}{\AFESBrandIllustration{lighthouse}} &
    magnifyingglass & \raisebox{-0.5\height}{\AFESBrandIllustration{magnifyingglass}} \\
    mountain & \raisebox{-0.5\height}{\AFESBrandIllustration{mountain}} &
    shakkas & \raisebox{-0.5\height}{\AFESBrandIllustration{shakkas}} \\
    \bottomrule
  \end{tabular}
\end{table}

The following \meta{options} are supported:

\begin{description}
  \item[bgcolour] The background colour. Default: |neutral|
  \item[colour] The foreground colour. Default: |hero|
  \item[height] The height as a dimension expression.
  \item[width] The width as a dimension expression.
\end{description}

If both |height| and |width| are specified the logo will be scaled to fit
within these dimensions.

These options can also be set using \cs{AFESBrandSetOptions} and the prefix
|illustration/|.

\subsubsection{Adding New Illustrations}

Adding new illustrations is a bit of a pain. The are implemented as two PDFs
with colour commands removed so that the colour can be set by \LaTeX{}.

The background PDF must be named |afesbrand_illustration_<name>_bg.pdf|. The
foreground PDF must be named |afesbrand_illustration_<name>_fg.pdf|. Then
|<name>| can be passed to \cs{AFESBrandIllustration} to include the
illustration.

I suggest creating the two PDFs as black and white illustrations in Inkscape
as a single path which is filled with RGB black but not stroked. See Figure
\ref{newillustrations}

\begin{figure}
  \centering
  \includegraphics[width=0.45\linewidth]{croissant_bg_inkscape.png}\quad
  \includegraphics[width=0.45\linewidth]{croissant_fg_inkscape.png}
  \caption{Creating new illustrations in Inkscape}\label{newillustrations}
\end{figure}

Make sure the two documents use the same bounding box so that they will
overlay correctly.

Once you've created the two PDF files you can edit the two files to replace 
|0 0 0 rg| with spaces (the file size and offsets must remain exactly the
same) in a text editor or use the supplied shell script
|illustration_remove_colour.sh| to remove the colour commands:

\begin{verbatim}
  $ ./illustration_remove_colour.sh
  Usage: illustration_remove_colour.sh <name>

  The current directory should contain the background and foreground
  PDFs named 'afesbrand_illustration_<name>_bg.pdf' and
  'afesbrand_illustration_<name>_fg.pdf' respectively.

  These two PDFs should be single colour RGB black illustrations
  only containing filled (not stroked) paths.
\end{verbatim}

\subsection{Hero Graphic}

\begin{function}{\AFESBrandHeroGraphic}
  \begin{syntax}
    \oarg{options}\marg{image}
  \end{syntax}
  Insert \meta{image} using the AFES branding hero graphic with specified
  \meta{options}. The image is initially sized to fit the width and height of
  the graphic, but can be scaled and given an offset. See Figure
  \ref{herographic} for example usage.
\end{function}

\begin{figure}
  \begin{lstlisting}
\colorbox{hero}{\AFESBrandHeroGraphic{herbert.png}}
  \end{lstlisting}

  \centering\setlength{\fboxsep}{5mm}%
  \colorbox{hero}{\AFESBrandHeroGraphic[width=3cm]{herbert.png}}
  \caption{Example Hero Graphic}\label{herographic}
\end{figure}

The following \meta{options} are supported:

\begin{description}
  \item[colour] The colour of the hero graphic. Default: |accent|
  \item[image/dx] $x$-offset of the image from the centre of the frame as a
    percentage of the width of the frame.
  \item[image/dy] $y$-offset of the image from the centre of the frame as a
    percentage of the hight of the frame.
  \item[image/scale] Scale factor of the image as a percentage. This is
    applied after the image has been resized to fit the width and height of
    the frame.
  \item[height] Height of the graphic (including the image) as a dimension
    expression.
  \item[width] Width of the graphic (including the image) as a dimension
    expression.
\end{description}

These options can also be set using \cs{AFESBrandSetOptions} and the prefix
|hero/|.

\subsection{Brand Mark}

\begin{function}{\AFESBrandMark}
  \begin{syntax}
    \marg{options}
  \end{syntax}
  Print the AFES Brand Mark with the specified \meta{options}.
\end{function}

See Figure \ref{brandmark} for example usage.

\begin{figure}
  \begin{lstlisting}
\AFESBrandMark{height=2.5cm, padding=none, left=UNI, top=MEET,
  right=JESUS}\quad
\AFESBrandMark{height=2.5cm, padding=none, centre=UNI, bottom=MEET\\
  JESUS}\quad
\AFESBrandMark{height=2.5cm, padding=none, bottom=\AFESBrandUniMeetJesus}
  \end{lstlisting}

  \centering
  \AFESBrandMark{height=2.5cm, padding=none, left=UNI, top=MEET,
    right=JESUS}\quad
  \AFESBrandMark{height=2.5cm, padding=none, center=UNI, bottom=MEET\\
    JESUS}\quad
  \AFESBrandMark{height=2.5cm, padding=none, bottom=\AFESBrandUniMeetJesus}
  \caption{Example brand mark usage}\label{brandmark}
\end{figure}

The following \meta{options} are supported:

\begin{description}
  \item[bgcolour] Set the background colour of the brand mark. Default: |none|
  \item[bottom] Text at the bottom of the brand mark.
  \item[centre] Text in the centre of the brand mark.
  \item[colour] Set the colour of the brand mark. Default: |hero|
  \item[height] Set the height of the brand mark (including any bottom text
    and |padding|). If both |height| and |width| are specified the logo will
    be scaled to fit within these dimensions.
  \item[left] Text on the left of the brand mark.
  \item[padding] Set the padding around the brand mark. The following values
    are supported:
    \begin{description}
      \item[anchored] The brand mark is horizontally centred with padding
        equal to twice the height of the top of the cross and with no padding
        on the bottom.
      \item[centred] The brand mark is centred with padding equal to twice the
        height of the top of the cross. This is the default.
      \item[none] The brand mark is printed without padding.
    \end{description}
  \item[right] Text on the right of the brand mark.
  \item[top] Text on the top of the brand mark.
  \item[width] Set the width of the brand mark (including any bottom text and
    |padding|). If both |height| and |width| are specified the logo will be
    scaled to fit within these dimensions.
\end{description}

These options can also be set using \cs{AFESBrandSetOptions} and the prefix
|mark/|.

\subsection{Wave Effect}

\begin{function}{\AFESBrandWaveEffect}
  \begin{syntax}
    \oarg{options}\marg{text}
  \end{syntax}
  Print \meta{text} using the AFES brand wave effect with the specified
  \meta{options}.
\end{function}

The following \meta{options} are supported:

\begin{description}
  \item[bend] The percentage of the \meta{text} height to apply as a bend at
    its vertical centre. Default: |5|
  \item[colour] The colour of the \meta{text}. Default: \emph{Current colour}.
  \item[simple] Create simple wave rather than the default complex wave which
    is horizontal on the top and bottom of \meta{text}. See Figure
    \ref{simplewave}.

    \begin{figure}
      \begin{lstlisting}
\AFESBrandWaveEffect[simple, style=a, size=\huge, bend=4, width=4em,
  colour=royalblue]{semester one 2025 events}
      \end{lstlisting}

      \centering
      \AFESBrandWaveEffect[simple, style=a, size=\huge, bend=4, width=4em,
        colour=royalblue]{semester one 2025 events}
      \caption{Simple wave example}\label{simplewave}
    \end{figure}

  \item[style] Select one of the three specified wave styles, |a|, |b| and
    |c|. See Figure \ref{standwaves}. Note that for these styles |bend| is set
    to |3.5| to match the style guide.

    \begin{figure}
      \begin{lstlisting}
\AFESBrandWaveEffect[style=a, colour=tangerineorange,
  size=\huge]{There is one who gives meaning to it all}\quad
\AFESBrandWaveEffect[style=b, colour=royalblue, width=5em,
  size=\huge]{There is one who gives meaning to it all}\quad
\AFESBrandWaveEffect[style=c, colour=huntergreen, width=5em,
  size=\huge]{There is one who gives meaning to it all}
    \end{lstlisting}

      \centering
      \AFESBrandWaveEffect[style=a, colour=tangerineorange,
        size=\huge]{There is one who gives meaning to it all}\quad
      \AFESBrandWaveEffect[style=b, colour=royalblue, width=5em,
        size=\huge]{There is one who gives meaning to it all}\quad
      \AFESBrandWaveEffect[style=c, colour=huntergreen, width=5em,
        size=\huge]{There is one who gives meaning to it all}
      \caption{Standard wave examples}\label{standwaves}
    \end{figure}

  \item[uppercase] Boolean controlling whether to transform \meta{text} to
    uppercase.
  \item[width] Set the width of the box that \meta{text} is set in.
\end{description}

These options can also be set using \cs{AFESBrandSetOptions} and the prefix
|wave/|.

The options supported by \cs{AFESBrandFontSelect} can also be used with or
without the |font/| prefix to style the font of the wave.

\subsection{Frame Device}

\begin{function}{\AFESBrandFrameDevice}
  \begin{syntax}
    \oarg{options}\marg{text}
  \end{syntax}
  Print the AFES brand frame device with the specified \meta{options}.
\end{function}

See Figure \ref{framedevice}.

\begin{figure}
  \begin{lstlisting}
\colorbox{tangerineorange}{%
  \AFESBrandFrameDevice[action=JOIN A BIBLE STUDY, colour=lightorange,
    size=\huge, wave/colour=white, style=a]{Make friends, meet Jesus.}}
  \end{lstlisting}

  {\centering\setlength{\fboxsep}{5mm}%
     \colorbox{tangerineorange}{%
       \AFESBrandFrameDevice[action=JOIN A BIBLE STUDY, colour=lightorange,
         size=\huge, wave/colour=white, style=a]{%
           Make friends, meet Jesus.}}\par}
  \caption{Example AFES brand frame device}\label{framedevice}
\end{figure}

\begin{description}
  \item[action] An alias for |action/text|.
  \item[action/text] The call to action text.
  \item[action/width] The width of the box that the call to action text is set
    in. Default: |3.5em|
  \item[colour] The colour of the frame. Default: |accent|
  \item[ldx] A dimension expression specifying the $x$ offset of the lower
    frame from it's default position. Default: |0pt|
  \item[ldy] A dimension expression specifying the $y$ offset of the lower
    frame from it's default position. Default: |0pt|
  \item[udx] A dimension expression specifying the $x$ offset of the upper
    frame from it's default position. Default: |0pt|
  \item[udy] A dimension expression specifying the $y$ offset of the upper
    frame from it's default position. Default: |0pt|
\end{description}

These options can also be set using \cs{AFESBrandSetOptions} and the prefix
|frame/|.

The options supported by \cs{AFESBrandWaveEffect} and \cs{AFESBrandFontSelect}
can also be used to style the wave without or without the |wave/| and |font/|
prefixes. Note that where options clash (such as |colour|) then you need to
specify |wave/colour|.

\begin{function}{\AFESBrandFrameImage}
  \begin{syntax}
    \oarg{options}\marg{image}
  \end{syntax}
  Print the AFES brand frame device around \meta{image} with the specified
  \meta{options}.
\end{function}

{\centering\AFESBrandFrameImage[width=3cm]{mallsballs.jpg}\par}

The following \meta{options} are supported:

\begin{description}
  \item[bgcolour] Set the background colour of the image frame. Default:
    |hero|
  \item[colour] Set the colour of the image frame. Default: |accent|
  \item[height] Set the height of the image frame (including any |padding|).
    If both |height| and |width| are specified the image frame will be scaled
    to fit within these dimensions.
  \item[padding] Set the padding around the image frame. The following values
    are supported:
    \begin{description}
      \item[anchored] The image frame is horizontally centred with padding
        equal to twice the height of the top of the cross and with no padding
        on the bottom.
      \item[centred] The image frame is centred with padding equal to twice
        the height of the top of the cross. This is the default.
      \item[none] The image frame is printed without padding.
    \end{description}
  \item[width] Set the width of the image frame (including any |padding|).  If
    both |height| and |width| are specified the image frame will be scaled to
    fit within these dimensions.
\end{description}

\section{National Applications}

\subsection{Grid System}

\begin{function}{\AFESBrandGridSystem}
  \begin{syntax}
    \oarg{options}\marg{text}
  \end{syntax}
  Print and AFES brand grid system with the specified \meta{options}
  containing \meta{text} in the body area.

  See Figure \ref{gridsystem}.
\end{function}

\begin{figure}
  \begin{lstlisting}
\colorbox{neutral}{\AFESBrandGridSystem[body/colour=hero,
  head/colour=white]{\AFESBrandFontHeadOne\color{accent}COMMS\\
  \textcolor{white}{TITLE GOES}\\ HERE}}
  \end{lstlisting}
  
  \setlength{\fboxsep}{5mm}%
  \colorbox{neutral}{\AFESBrandGridSystem[body/colour=hero,
    head/colour=white, head/width=\dimexpr\linewidth-1cm]{%
    \AFESBrandFontHeadOne\color{accent}COMMS\\
    \textcolor{white}{TITLE GOES}\\ HERE}}
  \caption{Example AFES brand grid system}\label{gridsystem}
\end{figure}

The following \meta{options} are supported:

\begin{description}
  \item[body] A \meta{boolean} value specifying whether to include the body
    (and sidebar) or not. Default: |true|
  \item[body/colour] The colour of the body. Default: |neutral|
\item[body/height] The height of the body. Default: \emph{Natural height of
  content}
  \item[head/colour] The colour of the header. Default: |hero|
  \item[head/height] The height of the header. Default: |2cm|
  \item[head/width] The width of the header. Default: |\linewidth|
  \item[logo] The \meta{logo} to use. This must be a valid logo name which is
    passed to \cs{AFESBrandLogo}. Default: |national|

    Other values to control the logo can be passed using the |logo/| prefix
    and it's supported options. The defaults are:

    \begin{tabular}{@{}>{\ttfamily}l@{}>{\ttfamily}l@{}}
      logo/colour & = body/colour \\
      logo/height & = head/height \\
      logo/padding~ & = anchored \\
      logo/variant~ & = primarytagline \\
    \end{tabular}
      
  \item[sidebar] A \meta{boolean} value specifying whether to include the
    sidebar or note. Default: |true|
  \item[sidebar/colour] The colour of the sidebar. Default: |accent|
  \item[url] A \meta{boolean} value specifying whether to include the AFES URL
    in the header. Default: |true|
  \item[url/colour] The colour of the AFES URL. Default: |body/colour|
\end{description}

These options can also be set using \cs{AFESBrandSetOptions} and the prefix
|grid/|.

\subsection{Email Signature}

\begin{function}{\AFESBrandEmailSignature}
  \begin{syntax}
    \marg{options}
  \end{syntax}
  Create an AFES branded email signature block with the specified
  \meta{options}.
\end{function}

If \meta{options} is left empty, then a national style template block is
created. See Figure \ref{nationalemail}. A simplified local style is shown in
Figure \ref{localemail}.

\begin{figure}
  \begin{lstlisting}
\AFESBrandEmailSignature{}
  \end{lstlisting}
  \centering
  \AFESBrandEmailSignature{width=100mm, height=0mm}
  \caption{National email signature template.}\label{nationalemail}
\end{figure}

\begin{figure}
  \begin{lstlisting}
\AFESBrandSetColours{hero = tangerineorange}
\AFESBrandEmailSignature{
  local ,
  logo   = focus ,
  name   = David Purton ,
  role   = FOCUS Staff Team Leader ,
  campus = Adelaide University ,
  days   = {Tuesday, Wednesday and Thursday} ,
  phone  = 0413 626 862 ,
  web    = www.focusadelaide.org.au
}
  \end{lstlisting}
  \centering
  \AFESBrandSetColours{hero = tangerineorange}
  \AFESBrandEmailSignature{
    local ,
    logo   = focus ,
    name   = David Purton ,
    role   = FOCUS Staff Team Leader ,
    campus = Adelaide University ,
    days   = {Tuesday, Wednesday and Thursday} ,
    phone  = 0413 626 862 ,
    web    = www.focusadelaide.org.au ,
    width  = 100mm ,
    height = 0mm
  }
  \caption{Local email signature example.}\label{localemail}
\end{figure}

The following \meta{options} are supported:

\begin{description}
  \item[abn] Set the ABN. Default: |91 509 626 599|
  \item[acn] Set the ACN. Default: |145 358 185|
  \item[campus] Set your campus. Default: |Your Campus|
  \item[days] Set you work days. Default: \emph{empty}

    Example: |days = {Tuesday, Wednesday and Thursday}|. Don't forget to
    protect any commas with braces.

  \item[height] Set the height of the signature block. Default: |10.583mm|

    The signature block will be resized to this height or less. The value of
    |10.583mm| translates to |125px| at 300dpi for easy conversion to bitmap.
    If a value of |0pt| is given, the natural height based off the |width|
    option will be used. (If both |width| and |height| are |0pt|, the
    signature block is sized based off a |13ex| high logo.)

  \item[local] Clear the default national options (|abn|, |acn|, |phone|,
    |postal| and |web|). This should be placed as the first option. Local
    values for |phone|, |web|, etc.\ can then be added.
  \item[logo] Set the main logo name. Default: |national|

    By default the |primarytagline| variant will be selected. Complete control
    of the logo is available by passing supported logo options with the
    |logo/| prefix.

  \item[name] Set your name. Default: |Your Name|
  \item[phone] Set your phone number. Default: |(02) 9697 0313|
  \item[postal] Set your postal address. Default: |PO Box 684\break Kingsford NSW 2032|
  \item[role] Set your role. Default: |Your Role|
  \item[web] Set the web site URL. Default: |www.afes.org.au|
  \item[width] Set the width of the signature block. Default: |50.8mm|

    The signature block will be resized to this width or less. The value of
    |50.8mm| translates to |600px| at 300dpi for easy conversion to bitmap.
    If a value of |0pt| is given, the natural width based off the |height|
    option will be used. (If both |width| and |height| are |0pt|, the
    signature block is sized based off a |13ex| high logo.)
\end{description}

Any of the field options can be left out of the signature block by setting
them to an empty token list, e.g., |campus=|.

These options can also be set using \cs{AFESBrandSetOptions} and the prefix
|email/|.

Convert the signature block to a PNG with an alpha channel, e.g., using
GhostScript. The default dimensions will produce a signature block with
maximum width 600px and maximum height 125px. This should be safe for most
email clients.

\begin{lstlisting}
/usr/bin/gs -dSAFER -dNOPAUSE -dBATCH -sOutputFile=email_signature.png \
  -sDEVICE=pngalpha -r300 -dTextAlphaBits=4 -dGraphicsAlphaBits=4 \
  email_signature.pdf
\end{lstlisting}


\section{Local Applications}

\subsection{Profile Picture}

\begin{function}{\AFESBrandProfile}
  \begin{syntax}
    \oarg{options}
  \end{syntax}
  Print a profile picture for use on social media. This uses the brand mark
  with appropriate padding.
\end{function}

\AFESBrandProfile[size=1.5cm]

The following \meta{options} are supported:

\begin{description}
  \item[bgcolour] Set the background colour of the profile picture. Default:
    |hero|
  \item[colour] Set the colour of the profile picture. Default: |accent|
  \item[size] Set the size of the profile picture.
\end{description}

These options can also be set using \cs{AFESBrandSetOptions} and the prefix
|profile/|.

\subsection{Social Media Banners and Posts}

\begin{function}{AFESBrandSocialPost}
  \begin{syntax}
    \oarg{options}\marg{platform}
  \end{syntax}
  Environment to create a social media post for \meta{platform} with specified
  \meta{options}.
\end{function}

Posts are created to size correctly when converted to a bitmap image at
300dpi. Content is wrapped in a |tikzpicture|, so the post should be
constructed with |tikz| \cs{node} and \cs{path} macros. The content is
clipped to the post size and the origin is at the centre of the background.

The best way to create a social post is to use the |standalone| class which
sizes the document to its content. Then convert the resulting PDF to a
bitmap at 300dpi. E.g., with GhostScript:

\begin{lstlisting}
/usr/bin/gs -dSAFER -dNOPAUSE -dBATCH -sOutputFile=post.png \
  -sDEVICE=png16m -r300 -dTextAlphaBits=4 -dGraphicsAlphaBits=4 \
  post.pdf
\end{lstlisting}

\medskip

See Figure \ref{instagram} for an example Instagram post.

\begin{figure}
  \begin{lstlisting}
\begin{AFESBrandSocialPost}[type=portrait]{instagram}
  \node{\AFESBrandWaveEffect[colour=white, size=30, style=b, bend=5,
    linespace=0.935, width=4em]{You're not just here to pass exams}};
  \node[xshift=2.25cm, yshift=-1.45cm]{%
    \AFESBrandIllustration[bgcolour=white, colour=tangerineorange,
    width=1.262cm]{lighthouse}};
\end{AFESBrandSocialPost}
  \end{lstlisting}
  \centering
  \begin{AFESBrandSocialPost}[type=portrait]{instagram}
    \node{\AFESBrandWaveEffect[colour=white, size=30, style=b, bend=5,
      linespace=0.935, width=4em]{You're not just here to pass exams}};
    \node[xshift=2.25cm, yshift=-1.45cm]{%
      \AFESBrandIllustration[bgcolour=white, colour=tangerineorange,
      width=1.262cm]{lighthouse}};
  \end{AFESBrandSocialPost}
  \caption{Example Instagram post}\label{instagram}
\end{figure}

The following values of \meta{platform} are defined:

\begin{description}
  \item[facebook] Create a Facebook post.

    The |type| option supports |banner|, |eventcover|, |groupcover|,
    |landscape|, |portrait| and |square|.

  \item[instagram] Create an Instagram post.

    The |type| option supports |landscape|, |portrait| and |square|.

  \item[trybooking] Create a Trybooking banner.

    Only |banner| is supported for the |type| option.
\end{description}

The following \meta{options} are supported:

\begin{description}
  \item[colour] The background colour of the social post. Default: |hero|
  \item[height] A custom height as a dimension expression. If |height| and
    |width| are set, these are used instead of the defined values for
    \meta{platform} and |type|.
  \item[type] The type of post. See above for defined values for each
    \meta{platform}. Default: |square|
  \item[width] A custom width as a dimension expression. If |height| and
    |width| are set, these are used instead of the defined values for
    \meta{platform} and |type|.
\end{description}

These options can also be set using \cs{AFESBrandSetOptions} and the prefix
|social/|.

\end{document}
