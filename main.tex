% TeX Program = lualatex

% AFES Unified Branding Factory for LaTeX
%
% (c) 2026 David Purton <david.purton@afes.org.au>

\documentclass{standalone}

\usepackage{afesbrand}

% Set your local colours
%
%   hero    = huntergreen | royalblue   | sunnyyellow   | tangerineorange
%   neutral = dustyblue   | slatenavy   | lemonyellow   | naturalgrey
%   accent  = sagegreen   | digitalpink | mustardyellow | lightorange

\AFESBrandSetColours{
    , hero    = huntergreen
    , neutral = naturalgrey
    , accent  = dustyblue
}

% Optionally set a page colour by uncommenting the following line (default is
% transparent)

% \pagecolor{white}

\begin{document}

% Set text in one of the three standard AFES wave styles
%
%   style = a | b | c
%
% Other common adjustments include adjusting the maximum width of the text
% block and the bend percentage. You can insert manual line breaks with \\ (a
% double backslash).
%
% See the documentation in afesbrand.pdf in the file list to the left of the
% screen for full options to customise the wave style (and instructions for
% producing other brand elements).

\AFESBrandWaveEffect[
    , style  = a
    , colour = hero
    , size   = 42
    , width  = 4em
    , bend   = 3.5
]{%
    There is one who gives meaning to it all
}

% The output can be downloaded as a PDF by clicking on the down arrow next to
% the 'Recompile' button up the top of the right hand panel.
%
% A PNG and an SVG is also available:
%   - Click on the 'Logs and output files' button next to the 'Recompile'
%     button.
%   - Click on 'Other logs and files' at the bottom right of the screen.
%   - Download 'AFESBrandFactory.png' or 'AFESBrandFactory.svg'

\end{document}
